\documentclass[12pt]{article}
\usepackage{amsmath}
\usepackage{amsthm}
\usepackage{amssymb}
\usepackage{enumitem}
\usepackage[margin=1in]{geometry}

\title{Process for Forward Fast Fourier Transform (FFT) Numerical Computation}
\author{}
\date{}

\begin{document}
\maketitle
\subsection*{Description}
Forward FFT takes time domain data points and converts them to frequency domain coefficients.

\subsection*{Process Overview}
Solve for complex numbers ($C_k$), then convert to $a,b$ coefficients for use in trigonometric interpolation equation. 

\section*{Prerequisites}
\begin{itemize}
    \item $N$ = Number of data points. $N$ must be a power of 2, can zero-pad otherwise. Data points must be evenly spaced on $[-\pi,\pi]$.
    \item $m$ = Degree. Higher solves for higher frequencies in the approximation. $m \leq N/2$
\end{itemize}

\section*{Step 1) Generate Data Points}
For data points $(x_j,y_j)$ on $[-\pi,\pi]$:
\[x_j = -\pi + \frac{2\pi j}{N}, \quad j=0,1,\ldots,N-1\]
\[y_j = f(x_j), \quad j=0,1,\ldots,N-1\]

\section*{Step 2) Pre-compute setup values:}
\subsection*{a) Pre-computed "Roots of Unity":}
\[W_k = e^{-2\pi ik/N} = \cos(2\pi k/N) - i\sin(2\pi k/N), \quad k=0,1,\ldots,(N/2)-1\]
\\
Note: Only need to compute each $W_k$ once.\\
Note: $W$ values "loop" around a unit circle for higher values.\\
\begin{align*}
\text{ex: } N=8&, \text{with } \log_2(8)= 3: \\
W_0 &= e^{-2\pi i(0)/8} = 1.0000+0.0000i &(0^\circ)\\
W_1 &= e^{-2\pi i(1)/8} = 0.7071-0.7071i &(45^\circ)\\
W_2 &= e^{-2\pi i(2)/8} = 0.0000-1.0000i &(90^\circ)\\
W_3 &= e^{-2\pi i(3)/8} = -0.7071-0.7071i &(135^\circ)\\
\cline{1-3}\\
W_4 &= -W_0 = -1.0000+0.0000i &(180^\circ)\\
W_5 &= -W_1 = -0.7071+0.7071i &(225^\circ)\\
...
\end{align*}

Loop rules:
\begin{itemize}
    \item $W_{k+4} = -W_k$ (opposite side of circle)
    \item $W_{k+8} = W_k$ (full rotation)
\end{itemize}

\subsection*{b) Initial complex values array:}
\begin{enumerate}[label=\roman*)]
    \item Set initial complex values to y values:
    \[C_j = y_j, \quad j=0,1,\ldots,N-1\]
    \item Bit-reverse the complex values index positions so butterfly pairing is correct.\\
    For array size $N$, each index $i$ (for 0 to $N-1$) is found by:
    \begin{enumerate}
        \item Convert to binary in $\log_2(N)$ bits
        \item Reverse the bits
        \item Convert back to decimal
        \item Re-order complex pairs with new index
    \end{enumerate}
\end{enumerate}

Ex: $N=8$, with $\log_2(8)=3$ bits:
\begin{center}
\begin{tabular}{|c|c|c|c|}
\hline
i & binary & reversed & new i\\
\hline
0 & 000 & 000 & 0\\
1 & 001 & 100 & 4\\
2 & 010 & 010 & 2\\
3 & 011 & 110 & 6\\
4 & 100 & 001 & 1\\
5 & 101 & 101 & 5\\
6 & 110 & 011 & 3\\
7 & 111 & 111 & 7\\
\hline
\end{tabular}
\end{center}

\[C_0 \to C_0, C_1 \to C_4, C_2 \to C_2, C_3 \to C_6, \text{ etc.}\]
Note: Only need to compute bit-reversal index table once.

\section*{Step 3) Butterfly Pairs}
Combine complex values in pairs over stages, updating values one step at a time, until the final complex values are fully computed. Pairs of $C_k$ are found by increasing the distance when grouping the arrays:
\begin{itemize}
    \item Stage Number: $s = 1,2,\ldots,\log_2(N)$
    \item Distance between pairs in stage: $d = 2^{s-1}$
\end{itemize}

$ $\\
For example, with $N=8$ and $\log_2(8)=3$ stages:
\begin{align*}
\text{Stage 1 (distance = 1):} & \quad [0],[1],[2],[3],[4],[5],[6],[7] \to \{(0,1),(2,3),(4,5),(6,7)\}\\
\text{Stage 2 (distance = 2):} & \quad [0,1],[2,3],[4,5],[6,7] \to \{(0,2),(1,3),(4,6),(5,7)\}\\
\text{Stage 3 (distance = 4):} & \quad [0,1,2,3],[4,5,6,7] \to \{(0,4),(1,5),(2,6),(3,7)\}
\end{align*}
$ $\\
For each pair, perform the following ($N$ = total \# of points, $d$ = distance between pairs):
\begin{align*}
\eta &= C_{k+d} \times W_{(k\times N/2d)}\\
C_k &= C_k + \eta\\
C_{k+d} &= C_k - \eta
\end{align*}

\section*{Step 4) Coefficient Extraction}
Convert complex values to $a_k$ and $b_k$ values:
\[a_0 = \frac{1}{N}Re(C_0)\]
\[a_k = \frac{2}{N}Re(C_k) \quad \quad b_k = \frac{2}{N}Im(C_k), \quad \text{for } k=1,2,\ldots,(N/2)-1\]
\[a_{N/2} = \frac{1}{N}Re(C_{N/2}), \quad \text{for } k = N/2\]

\section*{Step 5) Construct Fourier series approximation}
Interpolating Polynomial Equation for FFT:
\[S_m(x) = \frac{a_0 + a_m\cos(mx)}{2} + \sum_{k=1}^{m-1}(a_k\cos(kx) + b_k\sin(kx))\]

\end{document}